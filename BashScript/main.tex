%!TEX program = xelatex
%!TEX TS-program = xelatex
%!TEX encoding = UTF-8 Unicode
%-*-coding : UTF-8 -*-
%main.tex
%
\documentclass[12pt]{article}  %设置文章总排版格式
\usepackage{url,fontspec,xltxtra,xunicode,graphicx,xeCJK,listings,float}
\setCJKmainfont{WenQuanYi Micro Hei}     % 设置中文字体
\setCJKmonofont{WenQuanYi Micro Hei}
\setromanfont{Ubuntu Mono} % 设置英文字体
\newfontfamily\courier{Ubuntu Mono}
\lstset{linewidth=1.1\textwidth,
        numbers=left, %设置行号位置 
        basicstyle=\small\courier,
        numberstyle=\tiny\courier, %设置行号大小  
        keywordstyle=\color{blue}\courier, %设置关键字颜色  
        %identifierstyle=\bf,
        commentstyle=\it\color[cmyk]{1,0,1,0}\courier, %设置注释颜色 
        stringstyle=\it\color[RGB]{128,0,0}\courier,
        %framexleftmargin=10mm,
        frame=single, %设置边框格式  
        %backgroundcolor=\color[RGB]{0.1,0.1,0.76},
        %escapeinside=``, %逃逸字符(1左面的键),用于显示中文  
        breaklines, %自动折行  
        extendedchars=false, %解决代码跨页时,章节标题,页眉等汉字不显示的问题  
        xleftmargin=2em,xrightmargin=2em, aboveskip=1em, %设置边距  
        tabsize=4, %设置tab空格数  
        showspaces=false %不显示空格  
        basicstyle=\small\courier
       } 
\title{Bash脚本}
\author{Mod233}
\date{\today}
\XeTeXlinebreaklocale "zh"					   % 
\XeTeXlinebreakskip = 0pt plus 1pt minus 0.1pt %自动换行
\bibliographystyle{plain} %参考文献的格式

\begin{document}   %正文开始

\maketitle
\begin{abstract}
sh脚本功能相对较弱,但对整个linux系统的通用性很强,编程过程也相对简单,并可以帮助日常linux的使用,决定开一篇来记录下,主要以具体的例子切入,遇到什么知识点就梳理什么知识点。\par
学习的网站:http://tldp.org/LDP/abs/html/comparison-ops.html\#EX13
\end{abstract}
\tableofcontents
\section{Bash 基本脚本} % (fold)
\ \ \ \ \ 这个章节主要讲一些简单shell命令组成的shell脚本,是很基础的东西,但经过几层嵌套后,就能够实现相对复杂场景。
\label{sec:熟悉latex}
\subsection{Other Comparsion Operators}
\ \ \ 举的第一个例子是字符串的比对,具体例子如下:
\begin{lstlisting}[language=sh]
#!/bin/bash
#  str-test.sh: Testing null strings and unquoted strings,
#+ but not strings and sealing wax, not to mention cabbages and kings . . .

# Using   if [ ... ]

# If a string has not been initialized, it has no defined value.
# This state is called "null" (not the same as zero!).

if [ -n $string1 ]    # string1 has not been declared or initialized.
then
  echo "String \"string1\" is not null."
else  
  echo "String \"string1\" is null."
fi                    # Wrong result.
# Shows $string1 as not null, although it was not initialized.

echo

# Let's try it again.

if [ -n "$string1" ]  # This time, $string1 is quoted.
then
  echo "String \"string1\" is not null."
else  
  echo "String \"string1\" is null."
fi                    # Quote strings within test brackets!

echo

if [ $string1 ]       # This time, $string1 stands naked.
then
  echo "String \"string1\" is not null."
else  
  echo "String \"string1\" is null."
fi                    # This works fine.
# The [ ... ] test operator alone detects whether the string is null.
# However it is good practice to quote it (if [ "$string1" ]).
#
# As Stephane Chazelas points out,
#    if [ $string1 ]    has one argument, "]"
#    if [ "$string1" ]  has two arguments, the empty "$string1" and "]" 


echo


string1=initialized

if [ $string1 ]       # Again, $string1 stands unquoted.
then
  echo "String \"string1\" is not null."
else  
  echo "String \"string1\" is null."
fi                    # Again, gives correct result.
# Still, it is better to quote it ("$string1"), because . . .


string1="a = b"

if [ $string1 ]       # Again, $string1 stands unquoted.
then
  echo "String \"string1\" is not null."
else  
  echo "String \"string1\" is null."
fi                    # Not quoting "$string1" now gives wrong result!

exit 0   # Thank you, also, Florian Wisser, for the "heads-up".
\end{lstlisting}\par
给出输出结果:
\begin{lstlisting}[language=sh]
➜  sh_script ./exp7-6.sh
exp1
String "string1" is not null.
exp2
String "string1" is null.
exp3
String "string1" is zero.
exp4
String "string1" is not null.
exp5
String "string1" is not null.

\end{lstlisting}\par
这个例子就是想说明,如果要想测试一个字符串是否是null,最好使用 \emph{if [ "\$string1" ]}利用引号来把变量包裹起来。\par
sh脚本中,数值比较中,字符串和数字采用不同操作符,看下面的例子:
\begin{lstlisting}[language=sh]
#!/bin/bash

a=4
b=5

#  Here "a" and "b" can be treated either as integers or strings.
#  There is some blurring between the arithmetic and string comparisons,
#+ since Bash variables are not strongly typed.

#  Bash permits integer operations and comparisons on variables
#+ whose value consists of all-integer characters.
#  Caution advised, however.

echo

if [ "$a" -ne "$b" ]
then
  echo "$a is not equal to $b"
  echo "(arithmetic comparison)"
fi

echo

if [ "$a" != "$b" ]
then
  echo "$a is not equal to $b."
  echo "(string comparison)"
  #     "4"  != "5"
  # ASCII 52 != ASCII 53
fi

# In this particular instance, both "-ne" and "!=" work.

echo

exit 0
\end{lstlisting}\par
输出结果:
\begin{lstlisting}[language=sh]
exp1
./exp7-5.sh: line 6: [: ne: binary operator expected
a is equal to b
exp2
a is not equal to b

\end{lstlisting}\par
这个例子就是想强调,在进行比较操作时,一般整数比较用 \emph{-eq, -ns, -gt.etc}等字符,而字符串一般用\emph{=,==,>,<}等符号。\par
部分时候,shell脚本需要对字符串进行操作,比如下面:
\begin{lstlisting}[language=sh]
#!/bin/bash
# zmore

# View gzipped files with 'more' filter.

E_NOARGS=85
E_NOTFOUND=86
E_NOTGZIP=87

if [ $# -eq 0 ] # same effect as:  if [ -z "$1" ]
# $1 can exist, but be empty:  zmore "" arg2 arg3
then
  echo "Usage: `basename $0` filename" >&2
  # Error message to stderr.
  exit $E_NOARGS
  # Returns 85 as exit status of script (error code).
fi  

filename=$1

if [ ! -f "$filename" ]   # Quoting $filename allows for possible spaces.
then
  echo "File $filename not found!" >&2   # Error message to stderr.
  exit $E_NOTFOUND
fi  

if [ ${filename##*.} != "gz" ]
# Using bracket in variable substitution.
then
  echo "File $1 is not a gzipped file!"
  exit $E_NOTGZIP
fi  

zcat $1 | more

# Uses the 'more' filter.
# May substitute 'less' if desired.

exit $?   # Script returns exit status of pipe.
#  Actually "exit $?" is unnecessary, as the script will, in any case,
#+ return the exit status of the last command executed.
\end{lstlisting}\par
这个脚本的亮点很多,具体如下:\par
1. \$\#指代参数的个数,\$n指代第n个参数。\par
2. 27行中,\emph{\$\{filename##\*\.\}}这个地方对变量进行了截取,仅保留'.'后的内容,更多的语法参考:http://mywiki.wooledge.org/BashGuide/Parameters#Parameter\_Expansion\par
3. 脚本中涉及了返回码,这个以后尽可能的有吧,显得代码很规范。\par
\begin{lstlisting}[language=sh]

\end{lstlisting}
\begin{lstlisting}[language=sh]

\end{lstlisting}
\begin{lstlisting}[language=sh]

\end{lstlisting}

安装完成后,进入GDB:
\begin{lstlisting}[language=sh]
➜  ~ gdb
GNU gdb (Ubuntu 7.11.1-0ubuntu1~16.5) 7.11.1
Copyright (C) 2016 Free Software Foundation, Inc.
...SNIP...
[*] 8 commands could not be loaded, run `gef missing` to know why.
gef➤  gef missing
[*] Command `ropper` is missing, reason  →  Missing `ropper` package for Python3, install with: `pip3 install ropper`.
[*] Command `unicorn-emulate` is missing, reason  →  Missing `unicorn` package for Python3. Install with `pip3 install unicorn`.
[*] Command `capstone-disassemble` is missing, reason  →  Missing `capstone` package for Python3. Install with `pip3 install capstone`.
[*] Command `set-permission` is missing, reason  →  Missing `keystone-engine` package for Python3, install with: `pip3 install keystone-engine`.
[*] Command `assemble` is missing, reason  →  Missing `keystone-engine` package for Python3, install with: `pip3 install keystone-engine`.


\end{lstlisting}\par
可能还会出现别的python3包缺失的情况,如果利用pip能完成安装最好,不行的就需要手动安装了,手动测试了下还是比较顺利的:\par
\begin{lstlisting}[language=sh]
➜  .gdbplugins git clone https://github.com/unicorn-engine/unicorn.git 
➜  .gdbplugins cd unicorn/bindings/python      
➜  python git:(master) ls
build        sample_all.sh      sample_mips.py              shellcode.py
dist         sample_arm64eb.py  sample_network_auditing.py  unicorn
Makefile     sample_arm64.py    sample_sparc.py             unicorn.egg-info
MANIFEST.in  sample_armeb.py    sample_x86.py
prebuilt     sample_arm.py      setup.cfg
README.TXT   sample_m68k.py     setup.py
➜  python git:(master) sudo python3 setup.py install
\end{lstlisting}\par
\subsection{GDB基本命令}
\ \ \ GDB调试的过程中一定要以具体的代码,不断在实践中学习各种命令。首先是\emph{start}和\emph{run}命令:
\begin{lstlisting}[language=sh]
➜  factorial-test git:(master) ✗ g++ -g -o factorial factorial.c 
➜  factorial-test git:(master) ✗ ls
factorial  factorial.c
➜  factorial-test git:(master) ✗ gdb factorial 
...SNIP...
(gdb) run 1
Starting program: /home/csober/Documents/Github/gdb-test/cpp-file/factorial-test/factorial 1
factorial 1  = 1
[Inferior 1 (process 8533) exited normally]
(gdb) run 2
Starting program: /home/csober/Documents/Github/gdb-test/cpp-file/factorial-test/factorial 2
factorial 2  = 2
\end{lstlisting}\par
当程序中出现错误后,GDB中也会提示错误出现的位置,并且可以使用\emph{backtrace}回溯函数调用过程,虽然,gdb没有说为什么程序崩溃,但一般标注在函数头,都是子函数调用引起的错误,可以使用backtrace来查看活动函数递归调用的栈帧结果。\par
一般情况下,栈帧由返回地址、函数的参数及局部变量组成。利用backtrace追踪的时候,从当前停止或暂停的最顶部函数开始,向下直到main()函数:
\begin{lstlisting}[language=sh]
(gdb) run -1
Starting program: /home/csober/Documents/Github/gdb-test/cpp-file/factorial-test/factorial -1

Program received signal SIGSEGV, Segmentation fault.
0x00000000004005fe in factorial (
    n=<error reading variable: Cannot access memory at address 0x7fffff7fefec>) at factorial.c:3
3	int factorial(int n){

(gdb) backtrace
#0  0x00000000004005fe in factorial (
    n=<error reading variable: Cannot access memory at address 0x7fffff7fefec>) at factorial.c:3
#1  0x0000000000400622 in factorial (n=-174641) at factorial.c:6
#2  0x0000000000400622 in factorial (n=-174640) at factorial.c:6
...SNIP...
#174639 0x0000000000400622 in factorial (n=-3) at factorial.c:6
#174640 0x0000000000400622 in factorial (n=-2) at factorial.c:6
#174641 0x0000000000400622 in factorial (n=-1) at factorial.c:6
#174642 0x0000000000400688 in main (argc=2, argv=0x7fffffffda48) at factorial.c:16
\end{lstlisting}\par
像上面,如果由于频繁子函数调用引起的栈溢出,可以使用 backtrace -num 从内向外打印指定层数的栈结构:
\begin{lstlisting}[language=sh]
(gdb) backtrace -5
#174638 0x0000000000400622 in factorial (n=-4) at factorial.c:6
#174639 0x0000000000400622 in factorial (n=-3) at factorial.c:6
#174640 0x0000000000400622 in factorial (n=-2) at factorial.c:6
#174641 0x0000000000400622 in factorial (n=-1) at factorial.c:6
#174642 0x0000000000400688 in main (argc=2, argv=0x7fffffffda48) at factorial.c:16
\end{lstlisting}\par

如果需要输出执行过程中某个变量的值,可以使用\emph{print}命令,如果想长期追踪某个变量的值,可以使用\emph{display}命令:
\begin{lstlisting}[language=sh]
(gdb) list
1	#include <cstdio>
2	#include <cstdlib>
3	int factorial(int n){
4	    int result = 1;
5	    if(n==0) return 1;
6	    result=factorial(n-1)*n;
7	    return result;
8	}
9	int main(int argc,const char*argv[]){
10	    int n, result;
(gdb) backtrace
#0  factorial (n=-1) at factorial.c:4
#1  0x0000000000400688 in main (argc=2, argv=0x7fffffffda48) at factorial.c:16
(gdb) print n
$1 = -1
(gdb) display n
3: n = -10
(gdb) si
0x000000000040061d	6	    result=factorial(n-1)*n;
3: n = -10
(gdb) si
factorial (n=0) at factorial.c:3
3	int factorial(int n){
3: n = 0
\end{lstlisting}\par
display命令还有种重要的用法是 \emph{display /i \$pc} 显示下一行要执行的汇编代码,这在调试二进制文件中非常有用:
\begin{lstlisting}[language=sh]
(gdb) display /i $pc
4: x/i $pc
=> 0x4005f6 <factorial(int)>:	push   %rbp
(gdb) si
0x00000000004005f7	3	int factorial(int n){
3: n = 0
4: x/i $pc
=> 0x4005f7 <factorial(int)+1>:	mov    %rsp,%rbp
(gdb) si
0x00000000004005fa	3	int factorial(int n){
3: n = 0
4: x/i $pc
=> 0x4005fa <factorial(int)+4>:	sub    $0x20,%rsp
(gdb) si
0x00000000004005fe	3	int factorial(int n){
3: n = 0
4: x/i $pc
=> 0x4005fe <factorial(int)+8>:	mov    %edi,-0x14(%rbp)

# 如果需要删除某个display
(gdb) undisplay 3
(gdb) undisplay 4
\end{lstlisting}\par
还有一个重要的命令是 \emph{x(examine)} ,用来检查输出特定内存单位的值,命令格式为 \emph{x/nfu addr};\par
\emph{n} 指明需要连续检测多少单位;
\emph{f} 指明输出的格式:(‘x’, ‘d’, ‘u’, ‘o’, ‘t’, ‘a’, ‘c’, ‘f’, ‘s’),其中 'i' 代表 machine instruction, 'x'(hexadecimal)代表以十六进制输出;\par
\emph{u} 指明单位字节大小: b(Bytes), h(Halfwords two btyes), w(Words four bytes), g(Giant words eight bytes)\par
\emph{addr} 指明地址位置。\par 
\begin{lstlisting}[language=sh]
gef➤  x/8xb 0x7fffffffd990
0x7fffffffd990:	0x11	0x00	0x00	0x00	0x00	0x00	0x00	0x00
gef➤  x/8xb $rbp-0x24 
0x7fffffffd90c:	0x01	0x00	0x00	0x00	0x00	0x00	0x00	0x00
\end{lstlisting}
\subsection{GDB断点命令}
\ \ \ GDB中设置断点是最需要掌握的功能,这里我以C程序断点、地址断点、条件断点,分别举例:
\begin{lstlisting}[language=sh]
gef➤  break m
gef➤  break *0x4005a6
Breakpoint 1 at 0x4005a6: file first_fist.cpp, line 3.

\end{lstlisting}
\subsection{GDB调试内存}
\ \ \ 程序个别情况下的崩溃在于内存的使用,比如内存泄露、错误访问。内存的剖析工具有多种。最简单的是top命令,top命令非常易于实现,但当自己需要添加插装代码和做手工分析的时候需要当量工作,特别是当程序中有很多不同模块使用动态内存的时候。
\par
大部分的内存剖析工具,主要通过监视\emph{malloc()/new }在堆上分配的动态内存,因此大部分也被称为堆剖析工具(heap profiler),该类工具会记录动态内存分配的时间、由谁分配、大小以及何时由谁释放。在程序结束后,还会输出图标和日志文件,展示内存使用的细节,并使得内存更容易分配给最大的内存使用者。
\par
常见的工具有: AQtime(Windows)、Massif(Linux),下面详细介绍Massif。作为Valgrind工具套件的一部分,非常易于使用,并且能够生成内存使用报告,供对应的可视化软件(Massif-Visualizer)分析:
\begin{lstlisting}[language=sh]
➜  malloc-test git:(master) ✗ valgrind --tool=massif ./malloc i 100000 8
==9726== Massif, a heap profiler
==9726== Copyright (C) 2003-2015, and GNU GPL'd, by Nicholas Nethercote
==9726== Using Valgrind-3.11.0 and LibVEX; rerun with -h for copyright info
==9726== Command: ./malloc i 100000 8
==9726== 
before malloc:  hit return to continue
➜  malloc-test git:(master) ✗ ls
fflush.cpp  malloc  malloc.cpp  massif.out.9726

\end{lstlisting}
\section{Stack} % (fold)
\ \ \ Stack是程序中一段重要的内存空间,在进程运行时就被创建。Stack主要被用来存储函数的局部变量、环境变量,环境变量主要是能够帮助程序在各个函数间跳转的地址,像rbp,rsp等。\par
Stack是程序运行过程非常核心的部分,主要是由rsp(栈顶 TOP)和rbp(栈底 BASE)两个寄存器管理的,涉及到的操作是PUSH(压栈)和POP(弹栈);每执行一次压栈操作后,rsp值就减少一定值(在我64位的操作系统中是减8),每执行一次弹栈操作后,rsp的值就增加一定值。\par
\subsection{Stack与函数调用}
\label{sec:组织你的文本}
\ \ \ 为了尽可能完整的展示函数的调用过程,我使用下面的C代码进行分析:\par
\begin{lstlisting}[language=c]
 #include <stdio.h>
 #include <string.h>
 void test(int a1,int a2,int a3,int a4,int a5,int a6,int a7,int a8,int a9,int a10,int a11,int a12,int a13,int a14,int a15,int a16,int a17,int a18){
     char str[10];
     memcpy(str,                                                                 "abcdefghijklmnopqrstuvwx\
yz1234567890987654321\
ABCDEFGHIJKLMNOPQRSTU\
VWXYZSUCHANICEDAYYESA\
BOYCANDOEVERYTHINGFORAGIRL",40);
     int a=2;
     a++;
     printf("%d\n",a);
 }
 int main(int argc,const char*argv[]){
     test(1,2,3,4,5,6,7,8,9,10,11,12,13,14,15,16,17,18);
     return 0;
 }

\end{lstlisting}\par
在函数调用的过程中,一般先将参数压栈,然后将子函数运行完后的返回地址压栈,然后调用call指令,执行子函数,具体看如下代码:
\begin{lstlisting}[language=sh]
|           0x0040067c           pushq $0x12 ; .//first_fist.c:11 ; 18
|           0x0040067e           pushq $0x11 ; 17
|           0x00400680           pushq $0x10 ; 16
|           0x00400682           pushq $0xf ; 15
|           0x00400684           pushq $0xe ; 14
|           0x00400686           pushq $0xd ; 13
|           0x00400688           pushq $0xc ; 12
|           0x0040068a           pushq $0xb ; 11
|           0x0040068c           pushq $0xa ; 10
|           0x0040068e           pushq $9 ; 9
|           0x00400690           pushq $8 ; 8
|           0x00400692           pushq $7 ; 7
|           0x00400694           movl $6, %r9d
|           0x0040069a           movl $5, %r8d
|           0x004006a0           movl $4, %ecx
|           0x004006a5           movl $3, %edx
|           0x004006aa           movl $2, %esi
|           0x004006af           movl $1, %edi
|           0x004006b4           callq sym.test
|           0x004006b9           addq $0x60, %rsp ; '`'
|           0x004006bd           movl $0, %eax ; .//first_fist.c:12
|           0x004006c2           leave ; .//first_fist.c:13
|           0x004006c3           retq
\end{lstlisting}\par\par
上面的代码有几个细节值得关注:1. 子函数的参数,一般由 edi,esi,edx,ecx等六个寄存器存储,当参数超过六个后,再由push指令压入栈空间中。2. callq 指令是由 \emph{push eip; jmp addr}两条指令组合在一起的。\par
下面具体看下进行了jump跳转后,指令的执行情况:\par
\begin{lstlisting}[language=sh]
/ (fcn) sym.test 119
|   sym.test ();
|           ; var int local_48h @ rbp-0x48
|           ; var int local_44h @ rbp-0x44
|           ; var int local_40h @ rbp-0x40
|           ; var int local_3ch @ rbp-0x3c
|           ; var int local_38h @ rbp-0x38
|           ; var int local_34h @ rbp-0x34
|           ; var int local_24h @ rbp-0x24
|           ; var int local_20h @ rbp-0x20
|           ; var int local_8h @ rbp-0x8
|           0x004005f6           pushq %rbp ; .//first_fist.c:3
|           0x004005f7           movq %rsp, %rbp
|           0x004005fa           subq $0x50, %rsp ; 'P'
|           0x004005fe           movl %edi, local_34h
|           0x00400601           movl %esi, local_38h
|           0x00400604           movl %edx, local_3ch
|           0x00400607           movl %ecx, local_40h
|           0x0040060a           movl %r8d, local_44h
|           0x0040060e           movl %r9d, local_48h
|           0x00400612           movq %fs:0x28, %rax ; [0x28:8]=-1 ; '(' ; 40
|           0x0040061b           movq %rax, local_8h
|           0x0040061f           xorl %eax, %eax
|           0x00400621           leaq local_20h, %rax ; .//first_fist.c:5
|           0x00400625           movl $0x28, %edx ; '(' ; 40
|           0x0040062a           movl $str.abcdefghijklmnopqrstuvwxyz1234567890987
654321ABCDEFGHIJKLMNOPQRSTUVWXYZSUCHANICEDAYYESABOYCAND
OEVERYTHINGFORAGIRL, %esi ; 0x400758 ; 
"abcdefghijklmnopqrstuvwxyz1234567890987654321ABCDEFGHI
JKLMNOPQRSTUVWXYZSUCHANICEDAYYESABOYCANDOEVERYTHINGFORAGIRL"
|           0x0040062f           movq %rax, %rdi
|           0x00400632           callq sym.imp.memcpy ; void *memcpy(void *s1, const void *s2, size_t n)
|           0x00400637           movl $2, local_24h ; .//first_fist.c:6
|           0x0040063e           addl $1, local_24h ; .//first_fist.c:7
|           0x00400642           movl local_24h, %eax ; .//first_fist.c:8
|           0x00400645           movl %eax, %esi
|           0x00400647           movl $0x4007ca, %edi
|           0x0040064c           movl $0, %eax
|           0x00400651           callq sym.imp.printf ; int printf(const char *format)
|           0x00400656           nop ; .//first_fist.c:9
|           0x00400657           movq local_8h, %rax
|           0x0040065b           xorq %fs:0x28, %rax
|       ,=< 0x00400664           je   0x40066b
|       |   0x00400666           callq sym.imp.__stack_chk_fail ; void __stack_chk_fail(void)
|       `-> 0x0040066b           leave
\           0x0040066c           retq

\end{lstlisting}\par
从上面的代码中,也能得到许多重要的提示:\par
1. 执行完call命令后,子函数首先要将当前rbp寄存器值压栈。\par
2. 将rbp压栈后,对rbp赋新值,并将rsp减去一定值,抬高栈空间,这个数值是依据子函数所需求的内存来定的。\par
3. 对刚才通过 edi 等寄存器存储的参数,会存储到新开辟的栈空间中。\par
4. ret指令与call指令对应,实际是\emph{pop rip},然后rip执行下一条指令。\par
下面检测栈空间的情况:
\begin{lstlisting}[language=sh]
0x00007fffffffdac0│+0x00: 0x00007fffffffdb40  →  0x00000000004006d0  →  <__libc_csu_init+0> push r15	 ← $rsp, $rbp
0x00007fffffffdac8│+0x08: 0x00000000004006b9  →  <main+76> add rsp, 0x60
0x00007fffffffdad0│+0x10: 0x0000000000000007
0x00007fffffffdad8│+0x18: 0x0000000000000008
0x00007fffffffdae0│+0x20: 0x0000000000000009
0x00007fffffffdae8│+0x28: 0x000000000000000a
0x00007fffffffdaf0│+0x30: 0x000000000000000b
0x00007fffffffdaf8│+0x38: 0x000000000000000c
\end{lstlisting}\par
可以看到,栈空间中 \emph{0x00007fffffffdad0-0x00007fffffffdaf8}都是通过push指令压栈的参数,\emph{0x00007fffffffdac8} 位置存储是返回地址(即test函数执行完后的下一条指令)\par
今天在尝试栈溢出来覆盖返回地址的实验时,发现始终不能实现,在执行了\emph{test:memcpy} 函数后,指令直接返回了 \emph{main}函数当中,和学长的讨论中发现是由于stack-check的原因,可以从子函数的汇编代码中看到这样几条指令:
\begin{lstlisting}[language=sh]
|           0x00400612           movq %fs:0x28, %rax ; [0x28:8]=-1 ; '(' ; 40
|           0x0040061b           movq %rax, local_8h
|           0x0040061f           xorl %eax, %eax
...SNIP...
|       ,=< 0x00400664           je   0x40066b
|       |   0x00400666           callq sym.imp.__stack_chk_fail ; void __stack_chk_fail(void)
|       `-> 0x0040066b           leave
\end{lstlisting}\par
\%fs:0x28指向的是一个特殊地址,存储?????? \par
说明了Stack与子函数调用后,还想分享下今天尝试写shellcode遇到一些问题,测试的代码为:
\begin{lstlisting}[language=c]
#include <stdio.h>
#include <string.h>
void test(){
    char str[10];
    scanf("%s",&str);
    printf("%s\n",str);
}
int main(int argc,const char*argv[]){
    test();
    return 0;
}
\end{lstlisting}\par
为了方便调试,编译的时候使用-g加入的调试信息,并关闭了Stack-guard:
\begin{lstlisting}[language=sh]
➜  first_fist g++ -g scanf.c -fno-stack-protector -o scanf-nsp
\end{lstlisting}\par
输入数据为:\emph{123456789a123456789b123456789c123456789d},目的是想通过规则的数据来定位rip,通过内存检查,得到内存的布局为:
\begin{lstlisting}[language=sh]
0x00007fffffffd970│+0x00: 0x0000000000000000	 ← $rsp, $rsi
0x00007fffffffd978│+0x08: 0x0000000000000000
0x00007fffffffd980│+0x10: 0x00007fffffffd9a0  →  0x00000000004005b0  →  <__libc_csu_init+0> push r15	 ← $rbp
0x00007fffffffd988│+0x18: 0x00000000004005a7  →  <main+20> mov eax, 0x0
0x00007fffffffd990│+0x20: 0x00007fffffffda88  →  0x00007fffffffdeb9  →  "/home/csober/Documents/Github/how2stack/first_fist[...]"

...SNIP...

0x00007fffffffd970│+0x00: "123456789a123456789b123456789c123456789d"	 ← $rsp
0x00007fffffffd978│+0x08: "9a123456789b123456789c123456789d"
0x00007fffffffd980│+0x10: "789b123456789c123456789d"	 ← $rbp
0x00007fffffffd988│+0x18: "56789c123456789d"
0x00007fffffffd990│+0x20: "3456789d"
0x00007fffffffd998│+0x28: 0x0000000100000000
0x00007fffffffd9a0│+0x30: 0x00000000004005b0  →  <__libc_csu_init+0> push r15
0x00007fffffffd9a8│+0x38: 0x00007ffff7a2d830  →  <__libc_start_main+240> mov edi, eax


...SNIP...

gef➤  x /10xg 0x00007fffffffd960
0x7fffffffd960:	0x0000000000400470	0x0000000000400590
0x7fffffffd970:	0x3837363534333231	0x3635343332316139
0x7fffffffd980:	0x3433323162393837	0x3231633938373635
0x7fffffffd990:	0x6439383736353433	0x0000000100000000
0x7fffffffd9a0:	0x00000000004005b0	0x00007ffff7a2d830

\end{lstlisting}\par
从中可以看出,栈顶的位置为 0x7fffffffd970 ,当scanf读入数据后,返回地址和rbp都被覆盖了,但将 \emph{0x7fffffffd970-0x7fffffffd990} 地址段转换成ASCII码后,难以得到直观的存储结果:
\begin{lstlisting}[language=sh]
0x7fffffffd970:	8 7 6 5 4 3 2 1	-- 6 5 4 3 2 1 a 9
0x7fffffffd980:	4 3 2 1 b 9 8 7 -- 2 1 c 9 8 7 6 5
0x7fffffffd990: d 9 8 7 6 5 4 3
\end{lstlisting}\par
多方询问后,对结果解释如下:\par
1. 64位操作系统每次处理64位数据,从字节上体现的就是每次处理8个字节。 \par
2. x86\_64架构采用的是小端字节序,所以存储起来刚好相反。 \par
在这里要区分,指令push,pop对rsp进行操作时,与一般情况是反的,但数据存入内存中时,还是从低地址向高地址存储的。\par
为了直观完成返回地址覆盖的实验,我使用下面程序进行调试:
\begin{lstlisting}[language=sh]
 #include <stdio.h>
 #include <string.h>
 void test(){
     char str[10];
     scanf("%s",&str);
     printf("%s\n",str);
 }
 int main(int argc,const char*argv[]){
     test();
     printf("finish test\n");
     printf("tmp\n");
     printf("jump here\n");
     return 0;
 }
\end{lstlisting}\par
构建POC为:
\begin{lstlisting}[language=c]
import os

def print_unvisible():
    str = '12345678abcdefgh\xa0\xd9\xff\xff\xff\xf7\x00\x00\xbb\x05\x40\x00\x00\x00\x00\x00'
    print str

if __name__ == "__main__":
    print_unvisible()
\end{lstlisting}\par
GDB调试过程中,重定向了IO流:
\begin{lstlisting}[language=sh]
gef➤  run <input.txt 
Starting program: /home/csober/Documents/Github/how2stack/first_fist/scanf-nsp <input.txt
12345678abcdefgh������
jump here

Program received signal SIGBUS, Bus error.
\end{lstlisting}\par
最后提示\emph{Bus error},不知道具体原因,但实现了返回地址的覆盖,就暂时告一段落吧。\par
针对GDB调试重定向的问题,Github中给出一个有用的插件:https://github.com/Ovi3/pstdio,下次遇到必需的时候再来研究下。
\section{Heap} % (fold)
\label{sec:自动化工具}
\ \ \ Heap是程序运行过程中一段特殊的内存空间,Heap主要用于满足进程的动态内存请求。这一章节我会就Heap的类型、Linux下操作函数,及如何利用GDB来查看Heap空间的分配情况。
\subsection{Heap函数}
\ \ \ 堆的管理很复杂,但从上层C语言来看,主要由\emph{malloc}和\emph{free}进行管理。\par
\begin{quotation}
Malloc:void* malloc(size\_t size) \par
1. First time malloc is called in thread:\par
\ 1.1 Allocates a reasonable amount of memory\par
\ 1.2 Creates a heap segment or equivalent\par
\ 1.3 Return the caller a pointer to a memory region within it of suitable for demand.\par
2 Not the first call to malloc in thread:\par
\ 2.1 malloc will just return a pointer to a region within the current heap segment (or equivalent) of suitable size for the caller demand.\par \par \par

Free: void free(void *ptr)\par
1. Set a memory region previously returned from malloc as not in use.
\end{quotation}\par
尽管在各个系统架构下都有\emph{malloc}和\emph{free}两个函数对堆进行管理,但实现的细节可能很不相同,最常用的则是glibc库中提供的堆栈管理方法。整个章节的展开背景也是glibc。\par
在glibc中,涉及到的系统调用有/emph{int brk(void *addr),void *sbrin(void *addr), void* mmap(void *addr, size\_t length, int protect, int flags, int filedes, off\_t offset)}。\par
\emph{brk()和sbrk()}主要用于主线程申请内存空间,\emph{mmap}主要用在两种情况:1.申请的内存空间大于 \emph{DEFAULT\_MMAP\_THRESHOLD}; 2.子进程申请内存空间。
\subsection{Heap类型}
\ \ \ 在对glibc堆的工作流程进行深入学习前,需要对基本名词和堆类型有清晰的认识。\par
Arena: 是所有可被分配的内存,主要用于减少malloc调用与系统内核交互的次数,一般每个线程一个,并且个数是有限的。\par
Heap: 是一段连续的,由多个chunk组成的内存空间,每段heap只属于一个arena。\par
Chunk: 内存分配的最小单位,当进行malloc调用后,会返回chunk的可用段的地址,每个chunk仅存在于一个heap并且属于一个arena。\par
% \includegraphics[scale=•]{•}
在堆的管理中,还涉及到三个关键的数据结构:\par
1. malloc\_chunk: 管理组成一个chunk的所有内容\par
2. malloc\_info:  管理与进程中与堆交互的的所有内容\par
3. malloc\_state: 管理组成一个arena中的所有内容。\par
具体的数据结构如下:
\begin{lstlisting}[language=c]
struct malloc_chunk {
INTERNAL_SIZE_T prev_size; /* Size of previous chunk (if free). */
INTERNAL_SIZE_T size; /* Size in bytes, including header. */
struct malloc_chunk* fd; /* double links -- used only if free. */
struct malloc_chunk* bk;
/* Only used for large blocks: pointer to next larger size. */
struct malloc_chunk* fd_nextsize; /* double links -- used only if free. */
struct malloc_chunk* bk_nextsize;
};
\end{lstlisting}\par
可以看到,数据结构会依据内存是否占用而有所不同,
%\includegraphics[scale=•]{•}

\begin{lstlisting}[language=c]

\end{lstlisting}
\begin{lstlisting}[language=c]

\end{lstlisting}
\begin{lstlisting}[language=c]

\end{lstlisting}
\begin{lstlisting}[language=c]

\end{lstlisting}

\subsection{Heap工作流程}
\ \ \ 测试的时候使用的代码为:
\begin{lstlisting}[language=c]
#include <stdio.h>
#include <stdlib.h>
#include <string.h>
int main(){
	char* a = malloc(512);
	char* b = malloc(256);
	char* c;
    char* d = malloc(8);
    char* e = malloc(16);
    char* f = malloc(32);
    free(d);
    free(e);
    free(f);
	strcpy(a, "this is A!");
	free(a);
	c = malloc(500);
	strcpy(c, "this is C!");
}
\end{lstlisting}\par
将断点设置在\emph{malloc}函数后,观察几次申请堆空间过程中,chunk结构的变化:
\begin{lstlisting}[language=sh]
gef➤  heap chunks
[!] No heap section
gef➤  c
...SNIP...
gef➤  heap chunks
Chunk(addr=0x602010, size=0x210, flags=PREV_INUSE)
    [0x0000000000602010     00 00 00 00 00 00 00 00 00 00 00 00 00 00 00 00     ................]
Chunk(addr=0x602220, size=0x20df0, flags=PREV_INUSE)  ←  top chunk
gef➤  c
...SNIP...
Chunk(addr=0x602010, size=0x210, flags=PREV_INUSE)
    [0x0000000000602010     00 00 00 00 00 00 00 00 00 00 00 00 00 00 00 00     ................]
Chunk(addr=0x602220, size=0x110, flags=PREV_INUSE)
    [0x0000000000602220     00 00 00 00 00 00 00 00 00 00 00 00 00 00 00 00     ................]
Chunk(addr=0x602330, size=0x20ce0, flags=PREV_INUSE)  ←  top chunk
gef➤  c
...SNIP...
gef➤  heap chunks
Chunk(addr=0x602010, size=0x210, flags=PREV_INUSE)
    [0x0000000000602010     00 00 00 00 00 00 00 00 00 00 00 00 00 00 00 00     ................]
Chunk(addr=0x602220, size=0x110, flags=PREV_INUSE)
    [0x0000000000602220     00 00 00 00 00 00 00 00 00 00 00 00 00 00 00 00     ................]
Chunk(addr=0x602330, size=0x20, flags=PREV_INUSE)
    [0x0000000000602330     00 00 00 00 00 00 00 00 00 00 00 00 00 00 00 00     ................]
Chunk(addr=0x602350, size=0x20cc0, flags=PREV_INUSE)  ←  top chunk
gef➤  c
...SNIP...
gef➤  c
...SNIP...
gef➤  heap chunks
Chunk(addr=0x602010, size=0x210, flags=PREV_INUSE)
    [0x0000000000602010     00 00 00 00 00 00 00 00 00 00 00 00 00 00 00 00     ................]
Chunk(addr=0x602220, size=0x110, flags=PREV_INUSE)
    [0x0000000000602220     00 00 00 00 00 00 00 00 00 00 00 00 00 00 00 00     ................]
Chunk(addr=0x602330, size=0x20, flags=PREV_INUSE)
    [0x0000000000602330     00 00 00 00 00 00 00 00 00 00 00 00 00 00 00 00     ................]
Chunk(addr=0x602350, size=0x20, flags=PREV_INUSE)
    [0x0000000000602350     00 00 00 00 00 00 00 00 00 00 00 00 00 00 00 00     ................]
Chunk(addr=0x602370, size=0x30, flags=PREV_INUSE)
    [0x0000000000602370     00 00 00 00 00 00 00 00 00 00 00 00 00 00 00 00     ................]
Chunk(addr=0x6023a0, size=0x20c70, flags=PREV_INUSE)  ←  top chunk
Chunk(addr=0x602370, size=0x30, flags=PREV_INUSE)
Chunk size: 48 (0x30)
Usable size: 40 (0x28)
Previous chunk size: 0 (0x0)
PREV_INUSE flag: On
IS_MMAPPED flag: Off
NON_MAIN_ARENA flag: Off
gef➤  heap chunk 0x602350
Chunk(addr=0x602350, size=0x20, flags=PREV_INUSE)
Chunk size: 32 (0x20)
Usable size: 24 (0x18)
Previous chunk size: 0 (0x0)
PREV_INUSE flag: On
IS_MMAPPED flag: Off
NON_MAIN_ARENA flag: Off
gef➤  heap chunk 0x602010
Chunk(addr=0x602010, size=0x210, flags=PREV_INUSE)
Chunk size: 528 (0x210)
Usable size: 520 (0x208)
Previous chunk size: 0 (0x0)
PREV_INUSE flag: On
IS_MMAPPED flag: Off
NON_MAIN_ARENA flag: Off
\end{lstlisting}\par
从中可以发现几点:\par
1. 申请的内存空间在地址空间上是连续的。\par
2. 申请的每段空间都存在16字节对齐。\par
3. 申请的每段空间都有8字节用于标识。\par
下面查看具体内存段的载荷情况:
\begin{lstlisting}[language=sh]
gef➤  x /20xg 0x602328
0x602328:	0x0000000000000021	0x0000000000000000
0x602338:	0x0000000000000000	0x0000000000000000
0x602348:	0x0000000000000021	0x0000000000000000
0x602358:	0x0000000000000000	0x0000000000000000
0x602368:	0x0000000000000031	0x0000000000000000
0x602378:	0x0000000000000000	0x0000000000000000
0x602388:	0x0000000000000000	0x0000000000000000
\end{lstlisting}\par
可以看到0x21和0x31分表代表这段地址空间大小,及占用情况。目前都是申请了内存后,chunks的分配情况,当进行释放后:
\begin{lstlisting}[language=sh]
gef➤  c
...SNIP...
gef➤  c
...SNIP...
gef➤  x /40xg 0x602320
0x602320:	0x0000000000000000	0x0000000000000021
0x602330:	0x0000000000000000	0x0000000000000000
0x602340:	0x0000000000000000	0x0000000000000021
0x602350:	0x0000000000602320	0x0000000000000000
0x602360:	0x0000000000000000	0x0000000000000031
0x602370:	0x0000000000000000	0x0000000000000000
0x602380:	0x0000000000000000	0x0000000000000000
gef➤  c
...SNIP...
\end{lstlisting}\par
上面一个重要的点在于, \emph{0x602350}地址段,指明了上一个空的段的位置在\emph{0x602320},这样就和上面的知识串起来了。\par
检查bin链表的链接情况:
\begin{lstlisting}[language=sh]
gef➤  heap bin
─────────────────────[ Fastbins for arena 0x7ffff7dd1b20 ]─────────────────────
Fastbins[idx=0, size=0x10]  ←  Chunk(addr=0x602350, size=0x20, flags=PREV_INUSE)  ←  Chunk(addr=0x602330, size=0x20, flags=PREV_INUSE) 
Fastbins[idx=1, size=0x20]  ←  Chunk(addr=0x602370, size=0x30, flags=PREV_INUSE) 
\end{lstlisting}\par
当代码运行到释放了\emph{a}指针后:
\begin{lstlisting}[language=sh]
────────────────────[ Unsorted Bin for arena 'main_arena' ]────────────────────
[+] unsorted_bins[0]: fw=0x602000, bk=0x602000
 →   Chunk(addr=0x602010, size=0x210, flags=PREV_INUSE)
[+] Found 1 chunks in unsorted bin.
\end{lstlisting}\par
如果马上申请同样大小的内存段\emph{c},会发现\emph{Unsorted Bin}区的内存被从新分配,而不是从\emph{top chunk}中申请内存:
\begin{lstlisting}[language=sh]
────────────────────[ Unsorted Bin for arena 'main_arena' ]────────────────────
[+] Found 0 chunks in unsorted bin.

\end{lstlisting}\par

\subsection{Heap Overflow}
\ \ \ 这个小节介绍一些与堆溢出及利用有关的例子。
\begin{lstlisting}[language=sh]

\end{lstlisting}

\begin{lstlisting}[language=sh]
gef➤  heap chunks
Chunk(addr=0x602010, size=0x20, flags=PREV_INUSE)
    [0x0000000000602010     00 00 00 00 00 00 00 00 65 64 61 79 6f 00 00 00     ........edayo...]
Chunk(addr=0x602030, size=0x410, flags=PREV_INUSE)
    [0x0000000000602030     63 70 79 32 0a 68 20 6d 65 6d 63 70 79 31 0a 00     cpy2.h memcpy1..]
Chunk(addr=0x602440, size=0x20bd0, flags=PREV_INUSE)  ←  top chunk

\end{lstlisting}

\begin{lstlisting}[language=sh]

\end{lstlisting}


\section{Core} % (fold)
\ \ \ 这里的Core主要指的是程序运行过程意外退出,Linux系统自动生成的调试文件,可以供开发者回溯程序崩溃的原因。这个章节就来讲解Core Dumps的配置,及利用Core文件对堆栈、函数调用做分析。
\label{sec:玩转数学公式}
\subsection{Core配置}
\ \ \ 默认情况下,Core配置是关闭的,因为Core文件会拖慢程序崩溃后重启的速度,并设计敏感信息。通过\emph{ulimit \-c}命令和\emph{/procsys/kernel/core\_pattern}文件能够得到当前core文件的配置情况:\par
\begin{lstlisting}[language=sh]
➜  first_fist ulimit -c
0
➜  first_fist cat /proc/sys/kernel/core_pattern 
|/usr/share/apport/apport %p %s %c %d %P
\end{lstlisting}\par
下面先设置core文件大小,且将其以特定的命名规则存放在特定的文件目录中,在配置的时候可能会遇到权限不够,不能修改\emph{/proc/sys/kernel/core\_pattern}的情况,尤其是在ubuntu中,可以用\emph{sysctl}命令来实现:
\begin{lstlisting}[language=sh]
➜  ~ ulimit -c unlimited
➜  ~ sudo mkdir -p /var/cores
➜  ~ sudo echo "/var/cores/core.%e.%p.%g.%t" > /proc/sys/kernel/core_pattern
zsh: permission denied: /proc/sys/kernel/core_pattern
➜  ~ sudo sysctl -w kernel.core_pattern=/var/cores/core.%e.%p.%g.%t
[sudo] password for csober: 
kernel.core_pattern = /var/cores/core.%e.%p.%g.%t
➜  ~ cat /proc/sys/kernel/core_pattern 
/var/cores/core.%e.%p.%g.%t

\end{lstlisting}\par
如果之后想让core\_pattern的设置永久生效,可以在\emph{/etc/sysctl.conf}进行配置,具体配置之后再研究。特别说明下,ubuntu上是依靠Apport实现core dump信息的存储的,所以对emph{/proc/sys/kernel/core\_pattern}那步的配置不做,主要不知道具体怎么配置。配置好了core 的size之后,检查是否能得到core dump文件:
\begin{lstlisting}[language=sh]
➜  first_fist file core 
core: ELF 64-bit LSB core file x86-64, version 1 (SYSV), SVR4-style, from './scanf-nsp'
➜  first_fist gdb scanf-nsp core 
...SNIP...
Reading symbols from scanf-nsp...done.
[New LWP 17776]
Core was generated by `./scanf-nsp'.
Program terminated with signal SIGBUS, Bus error.
#0  main (argc=<error reading variable: Cannot access memory at address 0xf7ffffffd99c>, argv=<error reading variable: Cannot access memory at address 0xf7ffffffd990>) at scanf.c:14
14	}
\end{lstlisting}\par
最后的两句话给出了两点重要的提示,并且这些提示都是值得Google的:\par
1. 程序崩溃,并且系统发出了 SIGBUS,Bus error的提示。\par
2. 程序不能访问位于 0x7fffffffd99c 地址空间内容,并且错误的函数的位置在 scanf.c:14。\par
有时会遇到和动态库有关的错误提示,可以结合dpkt包管理器来定位错误包,如下:
\begin{lstlisting}[language=sh]
warning: JITed object file architecture unknown is not compatible with target architecture i386:x86-64.
Core was generated by `python ./cachetop.py'.
Program terminated with signal SIGSEGV, Segmentation fault.
#0  0x00007f0a37aac40d in doupdate () from /lib/x86_64-linux-gnu/libncursesw.so.5
# dpkg -l | grep libncursesw
ii  libncursesw5:amd64                  6.0+20160213-1ubuntu1                    amd64
     shared libraries for terminal handling (wide character support)
\end{lstlisting}
\subsection{Core 调试}
\ \ \ 完成了core的配置,并通过GDB进入后,下一步就是要提取core中的信息来查看程序崩溃时堆栈、函数调用的情况了,下面的例子我选用的都是别人的输出,因为别人选择的core报警相对复杂,可以看到更复杂的输出,一般步骤为:\par
1. 利用\emph{backtrace}命令回溯系统栈对函数的调用情况。\par
\begin{lstlisting}[language=sh]
(gdb) bt
#0  0x00007f0a37aac40d in doupdate () from /lib/x86_64-linux-gnu/libncursesw.so.5
#1  0x00007f0a37aa07e6 in wrefresh () from /lib/x86_64-linux-gnu/libncursesw.so.5
#2  0x00007f0a37a99616 in ?? () from /lib/x86_64-linux-gnu/libncursesw.so.5
#3  0x00007f0a37a9a325 in wgetch () from /lib/x86_64-linux-gnu/libncursesw.so.5
#4  0x00007f0a37cc6ec3 in ?? () from /usr/lib/python2.7/lib-dynload/_curses.x86_64-linux-gnu.so
#5  0x00000000004c4d5a in PyEval_EvalFrameEx ()
#6  0x00000000004c2e05 in PyEval_EvalCodeEx ()
#7  0x00000000004def08 in ?? ()
#8  0x00000000004b1153 in PyObject_Call ()
#9  0x00000000004c73ec in PyEval_EvalFrameEx ()
#10 0x00000000004c2e05 in PyEval_EvalCodeEx ()
#11 0x00000000004caf42 in PyEval_EvalFrameEx ()
#12 0x00000000004c2e05 in PyEval_EvalCodeEx ()
#13 0x00000000004c2ba9 in PyEval_EvalCode ()
#14 0x00000000004f20ef in ?? ()
#15 0x00000000004eca72 in PyRun_FileExFlags ()
#16 0x00000000004eb1f1 in PyRun_SimpleFileExFlags ()
#17 0x000000000049e18a in Py_Main ()
#18 0x00007f0a3be10830 in __libc_start_main (main=0x49daf0 <main>, argc=2, argv=0x7ffd33d94838, init=<optimized out>, fini=<optimized out>, rtld_fini=<optimized out>, 
    stack_end=0x7ffd33d94828) at ../csu/libc-start.c:291
#19 0x000000000049da19 in _start ()
\end{lstlisting}\par
\ \ \ 查看的函数栈的时候从下往上的顺序,如果中途出现 ”??“,一般是”symbol translation failed“。遇到这种情况时,可以找一些gdb的插件,或者在gcc编译的时候,保留符号信息(\emph{-fno-omit-frame-pointer -g})来修复这些问题。\par
\ \ \ 具体看下上面的输出,从frames 5 to 17 都是与python相关的调用,尽管不确定具体的modules调用情况,但基本的脉络为:\emph{wgetch()->wrefresh()->doupdate()},接下来就需要对栈中最顶层的 \emph{doupdate()} 进行分析。\par
2. 利用\emph{disas func}命令反汇编函数栈最上层函数。\par
\begin{lstlisting}[language=sh]
(gdb) disas doupdate
Dump of assembler code for function doupdate:
   0x00007f0a37aac2e0 <+0>:   push   %r15
   0x00007f0a37aac2e2 <+2>:   push   %r14
   0x00007f0a37aac2e4 <+4>:   push   %r13
   0x00007f0a37aac2e6 <+6>:   push   %r12
   0x00007f0a37aac2e8 <+8>:   push   %rbp
   0x00007f0a37aac2e9 <+9>:   push   %rbx
   0x00007f0a37aac2ea <+10>:  sub    $0xc8,%rsp
[...]
---Type <return> to continue, or q <return> to quit---
[...]
   0x00007f0a37aac3f7 <+279>: cmpb   $0x0,0x21(%rcx)
   0x00007f0a37aac3fb <+283>: je     0x7f0a37aacc3b <doupdate+2395>
   0x00007f0a37aac401 <+289>: mov    0x20cb68(%rip),%rax        # 0x7f0a37cb8f70
   0x00007f0a37aac408 <+296>: mov    (%rax),%rsi
   0x00007f0a37aac40b <+299>: xor    %eax,%eax
=> 0x00007f0a37aac40d <+301>: mov    0x10(%rsi),%rdi
   0x00007f0a37aac411 <+305>: cmpb   $0x0,0x1c(%rdi)
   0x00007f0a37aac415 <+309>: jne    0x7f0a37aac6f7 <doupdate+1047>
   0x00007f0a37aac41b <+315>: movswl 0x4(%rcx),%ecx
   0x00007f0a37aac41f <+319>: movswl 0x74(%rdx),%edi
   0x00007f0a37aac423 <+323>: mov    %rax,0x40(%rsp)
[...]

\end{lstlisting}
\ \ \ 只输入 \emph{disas} 命令也会默认的反汇编栈帧中最顶层的函数。标示”=>“代表出错执行的指令。根据这条指令,可以将错误定位到寄存器,下面查看寄存器的值即可。
3. 利用\emph{info registers}命令查看寄存器的值。\par
\begin{lstlisting}[language=sh]
(gdb) i r
rax            0x0  0
rbx            0x1993060    26816608
rcx            0x19902a0    26804896
rdx            0x19ce7d0    27060176
rsi            0x0  0
rdi            0x19ce7d0    27060176
rbp            0x7f0a3848eb10   0x7f0a3848eb10 <SP>
rsp            0x7ffd33d93c00   0x7ffd33d93c00
r8             0x7f0a37cb93e0   139681862489056
r9             0x0  0
r10            0x8  8
r11            0x202    514
r12            0x0  0
r13            0x0  0
r14            0x7f0a3848eb10   139681870703376
r15            0x19ce7d0    27060176
rip            0x7f0a37aac40d   0x7f0a37aac40d <doupdate+301>
eflags         0x10246  [ PF ZF IF RF ]
cs             0x33 51
ss             0x2b 43
ds             0x0  0
es             0x0  0
fs             0x0  0
gs             0x0  0
\end{lstlisting}
\ \ \ 可以看到, \%rsi的值为 0 ,很明显 0x0 不是一个有效的地址空间,出现了一种常见segfault:\emph{dereferencing an uninitialized or NULL pointer.}\par
4. 利用\emph{i proc m}命令检查内存分配情况。\par 
\begin{lstlisting}[language=sh]
(gdb) i proc m
Mapped address spaces:

      Start Addr           End Addr       Size     Offset objfile
        0x400000           0x6e7000   0x2e7000        0x0 /usr/bin/python2.7
        0x8e6000           0x8e8000     0x2000   0x2e6000 /usr/bin/python2.7
        0x8e8000           0x95f000    0x77000   0x2e8000 /usr/bin/python2.7
  0x7f0a37a8b000     0x7f0a37ab8000    0x2d000        0x0 /lib/x86_64-linux-gnu/libncursesw.so.5.9
  0x7f0a37ab8000     0x7f0a37cb8000   0x200000    0x2d000 /lib/x86_64-linux-gnu/libncursesw.so.5.9
  0x7f0a37cb8000     0x7f0a37cb9000     0x1000    0x2d000 /lib/x86_64-linux-gnu/libncursesw.so.5.9
  0x7f0a37cb9000     0x7f0a37cba000     0x1000    0x2e000 /lib/x86_64-linux-gnu/libncursesw.so.5.9
  0x7f0a37cba000     0x7f0a37ccd000    0x13000        0x0 /usr/lib/python2.7/lib-dynload/_curses.x86_64-linux-gnu.so
  0x7f0a37ccd000     0x7f0a37ecc000   0x1ff000    0x13000 /usr/lib/python2.7/lib-dynload/_curses.x86_64-linux-gnu.so
  0x7f0a37ecc000     0x7f0a37ecd000     0x1000    0x12000 /usr/lib/python2.7/lib-dynload/_curses.x86_64-linux-gnu.so
  0x7f0a37ecd000     0x7f0a37ecf000     0x2000    0x13000 /usr/lib/python2.7/lib-dynload/_curses.x86_64-linux-gnu.so
  0x7f0a38050000     0x7f0a38066000    0x16000        0x0 /lib/x86_64-linux-gnu/libgcc_s.so.1
  0x7f0a38066000     0x7f0a38265000   0x1ff000    0x16000 /lib/x86_64-linux-gnu/libgcc_s.so.1
  0x7f0a38265000     0x7f0a38266000     0x1000    0x15000 /lib/x86_64-linux-gnu/libgcc_s.so.1
  0x7f0a38266000     0x7f0a3828b000    0x25000        0x0 /lib/x86_64-linux-gnu/libtinfo.so.5.9
  0x7f0a3828b000     0x7f0a3848a000   0x1ff000    0x25000 /lib/x86_64-linux-gnu/libtinfo.so.5.9
[...]
\end{lstlisting}
\ \ \  从地址空间的分配能看到,\emph{0x400000-0x6e7000}是第一段有效内存空间,低于这个范围的,都是无效的。上面 \%rsi为0x0,就明显是一个无效的地址空间。

在ubuntu上,与core相关的说明:https://wiki.ubuntu.com/Apport
\section{绘制图表} % (fold)'[
\label{sec:绘制图表}
\section{幻灯片演示} % (fold)
\label{sec:幻灯片演示}
\section{从错误中救赎} % (fold)
\label{sec:从错误中救赎}
\section{问题探骊} % (fold)
\ \ \ 这里是我在学习软件调试过程遇到的一些问题,有些解决了,有些没有解决,就全部记在这里,供学习完作为思考题。
\label{sec:Latex无极限}
\subsection{动态链接库}
\ \ \ 1. 程序运行过程中,动态链接库是何时加载到内存空间中?\par
2. 动态链接库加载后,存储在内存空间中的哪里?\par
3.  
\subsection{Stack相关}
\ \ \ 1. memcpy 引起栈溢出,为什么不会影响memcpy函数?\par
2. memcpy 函数为什么不需要对rbp进行保存?\par
\subsection{Heap}
\ \ \ 1. malloc 申请了堆空间后,如何查看堆的位置?\par
2. 指针指向一个地址,那么结构体里面的函数代码地址怎么被确定的?
% section section_name (end)
\end{document} %正文结束
